\documentclass[a4paper, rebuttal, parskip=true, firsthead=false, fromemail=true, foldmarks=false]{scrlttr2}
\usepackage{amsmath}
\usepackage{amsfonts}
\usepackage{amssymb}
\usepackage[british]{babel}
\usepackage{hyperref}
%\address{Mathieu Leocmach and Hajime Tanaka,\\ Institute of Industrial Science,\\ University of Tokyo}
%\signature{Mathieu Leocmach and Hajime Tanaka} 
\begin{document} 
\begin{letter}{Dr. Helen Bache\\
Publishing Editor,\\
Royal Society of Chemistry\\
Thomas Graham House,\\
Milton Road,\\
Cambridge, UK. CB4 0WF}
\opening{\bf Dear Helen,}

Thank you very much for your e-mail concerning our manuscript (SM\nobreakdash- ART\nobreakdash-09\nobreakdash-2012\nobreakdash-027107) together with the comments of the Reviewers.

We are delighted by your positive response and we revised our manuscript in order to comply with the suggestions of both Reviewers.


We hope that you and your reviewers would find that the revised manuscript is now suitable for publication in Soft Matter. 

\closing{\bf Sincerely yours,} 
\clearpage

\textsf{\textbf{Replies to the comments of Referee 1}}

We thank the Reviewer for the time he devoted to the reading of our paper and his frank (positive) reply.

\textsf{\textbf{Replies to the comments of Referee 2}}

First, we thank the Reviewer for having carefully read our revised manuscript and provided useful comments to improve our manuscript. 
Hereafter we reply to the comments one by one.

\begin{quotationi}
1)      In the first paragraph, the authors state that ``algorithm that allow tracking of polydisperse particles in crowded environments have not reached the soft matter community.'' Contrary to this statement, however, there are indeed some work that have algorithms to deal with polydispersed objects, e.g. \url{http://www.pnas.org/content/107/31/13626.abstract} and \url{http://www.sciencemag.org/content/330/6001/197.abstract}. In those bacterial systems, their algorithm can track bacteria with different sizes as bacteria grow or move.
\end{quotationi}

Thank you for attracting our attention to these two references. However both algorythm rely on strong assumptions on the bacteria width. Zhang et al. use a $7\times 7$ pixels Hamming window to remove the image background, meaning that the maximum width they can detect is less that 7 pixels. Gibiansky et al. explicitely use the Crocker and Grier algorythm (their Reference S4). Both groups then exploit the anisotropy of the bacteria to pass the difficulty of the polydispersity in length.

In addition, bacteria polydispersity is not very high. The growth observed and quantified by Zhang et al. is the one of the clusters made of bacteria, not of the individual bacteria. The objects detected are the bacteria and the clusters are identified a posteriori from the positions and velocities of the bacteria.

Therefore we included these two references as an exemple of anisotropic object tracking that cannot claim real multiscale detection. We also precised that for anisotropic objects, the smallest dimension had to be almost monodisperse in the existing methods.

\begin{quotationi}
2)      How fast is this algorithm? How long does it take to finish one data set?
\end{quotationi}

\begin{quotationi}
3)      Since the real-to-optical size ratio changes with time, it will be better to show what the change is.
\end{quotationi}

\begin{quotationi}4) The authors state that ``We stress that this coupling of MRCO and icosahedral ordering with the spatial distribution of particle sizes does not imply fractionation, but rather \ldots'' However, we know that experimentally crystallization is suppressed when the polydispersity of colloidal hard spheres is above 12\% (Pusey, P. in Liquids, Freezing and Glass Transition (eds Hansen, J. P., Levesque, D. \& Zinn-Justin, J.) 763-931 (North-Holland, Amsterdam, 1991).). Since the polydispersity in this work is beyond 12\%, it seems to me that the system has to fractionize the particles by their sizes so that small nucleation clusters can start and grow in a local less polydispersed enviroment. This large polydispersity also slow down the dynamics of crystallization as more defects can be trapped between crystalline clusters.
\end{quotationi}

We hope that the Reviewer would think that the revised manuscript is now suitable for publication in Soft Matter. 


%\cc{Cclist} 
%\ps{adding a postscript} 
%\encl{list of enclosed material} 
\end{letter} 
\end{document}