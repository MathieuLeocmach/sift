\documentclass[preprint]{revtex4-1}
\usepackage{graphicx}
\usepackage{epstopdf}
\usepackage{amsmath}
\usepackage{hyperref}
\usepackage{booktabs}
\usepackage{color}

\usepackage{SIunits}
\newcommand{\wstar}{-0.023}
\newcommand{\Qstar}{0.25}

\setlength{\tabcolsep}{10pt}

\newenvironment{sistema}%
  {\left\lbrace\begin{array}{@{}l@{}}}%
  {\end{array}\right.}


\begin{document}
\title{Quantitative localisation and sizing of polydisperse colloids from confocal microscopy images}



\author{Mathieu Leocmach} 

\author{Hajime Tanaka}
\affiliation{ {Institute of Industrial Science, University of Tokyo, 4-6-1 Komaba, Meguro-ku, Tokyo 153-8505, Japan} }

\date{Received \today}

\begin{abstract}
We propose a scale-space based method to extract both the individual 3D coordinates and the radii of spherical colloids from confocal microscopy images. According to synthetic and real data, dilute or crowded, particles can be detected correctly without assumption on their sizes, even when particle diameters differ by large factors. Moreover the size of each particle can be estimated within a few percent error (less than $0.3\%$ if the diameter is larger than \unit{5}{px}).
\end{abstract}
\maketitle

Particle-level confocal microscopy experiments usually access the coordinates of the particles via the algorithm proposed by \citet{Crocker1996}. The original noisy image is blurred by convolution with a Gaussian kernel of width $\sigma$ to yield a soft peak per particle. Local intensity maxima within this blurred image give the coordinates of the particles with pixel resolution. Sub-pixel resolution ($0.1\sim0.3$~pixels error) can be achieved by taking the centre of mass of a neighbourhood around the local maxima. The extension of this algorithm in 3D has been done in two ways: either tracking particles in each confocal plane and reconstructing the results (2D-flavour)~\citep{dinsmore2001tdc}, or full image analysis on three dimensional pictures (3D-flavour)~\citep{vanblaaderen1995rss, Lu2007}.

The choice of the width $\sigma$ of the blurring kernel is critical: if it is too small, then the intensity profile is flat near the centre  of a particle, leading to multiple and ill-localized maxima per particle; if it is too large, then the peaks of nearby particles overlap, leading to shifts in the detected positions, or even fusion of the particles (only one particle detected instead of two). If the colloids are fairly monodisperse one can argue (at least in the 3D-flavour) that there exists a range of possible width where the choice of $\sigma$ has almost no effect on the number of particles detected. Choosing $\sigma$ within this range gives confidence in the localisation results.

\begin{figure*}
\begin{center}
\includegraphics{generate-figure0.pdf}
\end{center}
\caption{\textbf{Visualisation of the results of various tracking methods for the same portion of image.} \textbf{a,} Multiscale 3D tracking. \textbf{b,} Reconstruction from multiscale 2D tracking. \textbf{c-h,} Crocker and Grier in 3D with blurring radius increasing from \unit{2}{px} to \unit{4.5}{px} by steps of \unit{0.5}{px}. The circles on each picture are the result of 2D multiscale tracking of each XY slice of the 3D pictures. Sphere are displayed with radii determined by the tracking methods in \textbf{a-b}, and equal to the blurring radius for \textbf{c-h}.}
	\label{fig:localise}
\end{figure*}
\begin{figure*}
\begin{center}
\includegraphics{generate-figure1.pdf}
\end{center}
	\caption{\textbf{Sizing of our colloids.} \textbf{a,} Size distribution estimated \emph{in situ} (dashed line) by our multiscale algorithm ($\sim 1.7\times 10^6$ instantaneous sizing). Comparison with the size distribution estimated from \textsc{sem} of only $140$ dry particles (steps) is possible once $23\%$ of swelling of particle diameters is taken into account (full line). \textbf{b,} First peak of the radial distribution function with (full line) and without (dashed) the individual sizes data. Taking into account the measured sizes rectifies the effect of the polydispersity: the peak is thinner and higher.}
	\label{fig:sizing}
\end{figure*}

However, we found that no such ``good blur width'' exists in our polydisperse sample (see Fig.~\ref{fig:localise}c-h). The detection of smaller particles with small blurring widths leads to the failure in detecting properly the larger particles. This unacceptable failure of the \citet{Crocker1996} algorithm, as well as the want of the particles' radii data, triggered our design of a novel localisation algorithm that would be robust even for a system of polydispersity, which is unavoidable in real experiments. Here we briefly disclose our method, leaving full details to a future publication.

The key notion to detect objects of unknown and possibly diverse sizes in an image is the \emph{scale space}~\cite{Lindeberg1993}. A popular implementation for isotropic objects (or ``blobs'') is the Scale Invariant Feature Transform (\textsc{sift}) of \citet{Lowe2004}. It consists  in blurring the image by Gaussian kernels of logarithmically increasing widths ($\sigma_{i+1} = 2^{1/n} \sigma_i$, with $n$ a fixed integer, $n=3$ being a good choice) and taking the difference between consecutive blurred images. The difference of Gaussians (DoG) response function defined in this way depends on the position in space and on the scale. Bright objects in the original image are detected as local minima of the DoG in both space and scale, thus localisation and \emph{size} are determined simultaneously, without any assumption on the target size (see Fig.~\ref{fig:localise}a).

The \textsc{sift} is often used to match between different images from complex objects consisting of many rigidly linked blobs and further characterised by local histograms~\citep{Lowe2004}. To our knowledge, this method was never used for the quantitative localisation and sizing of independent single-blob objects like spherical colloids. The object-by-object optimal scale determination allows us to perform the spatial sub-pixel resolution step for each object on an image that is blurred just enough to have neither a flat intensity profile nor a nearby peak overlap. This leads to a spatial resolution below $0.3$~pixels when two $10$-pixels wide particles are at hard-core contact, and less than $0.03$~pixels error ($0.3\%$ of the diameter) when any other particle's surface is further than $1$~pixel from the surface of the particle of interest.

At infinite dilution the scale $\sigma$ is simply proportional to the radius $R$ of the particle. Assuming a binary ball object, one finds
\begin{equation}
	R = \sigma \sqrt{\frac{3 \log 2}{n(1-2^{-2/n})}}. 
	\label{eq:scale_dil}
\end{equation}
We found that the radius of an isolated pixelated ball can be indeed measured within $0.3\%$ relative error with this method, provided a sub-scale resolution step similar to the spatial sub-pixel resolution step.

In dense suspensions, the neighbouring particles influence the scale dependence of DoG response. This effectively shifts the minima of the DoG towards smaller scales, leading to smaller radii if one uses Eq.~(\ref{eq:scale_dil}) alone. Assuming once again binary ball objects, one can take this coupling into account and construct a $N\times N$ system, with $N$ the number of particles, whose solutions are the radii and whose coefficients depend on the inter-particle distances. This system is actually very sparse (less than two dozen non-zero coefficients per line, even in the dense suspensions studied here) and the results converges in about two iterations. In synthetic images we measured the error in the resulting radii to be around $1\%$. In an experimental image, the particles are not uniformly bright due to synthesis imperfection (quantity of dye fixed by each particle) and photo bleaching. If one does not take into account the relative brightness of the particles the less bright will appear smaller.

Figure~\ref{fig:sizing}a shows the size distribution of the suspensions investigated in the main text. Once a solvent swelling of $23\%$ in radius is taken into account, our \emph{in situ} measurements compare very well with the size distribution obtained from scanning electron microscopy (\textsc{sem}) of the same `dry' colloids. The polydispersity measured \emph{in situ} is $6.9\%$, where the polydispersity computed from \textsc{sem} is $6.2\%$.

The consistency of our method can be checked by constructing the radial distribution function $g(r)$. In monodisperse hard spheres, the $g(r)$ has a sharp first peak at $r=1\sigma$ corresponding to hard core contact. Polydispersity implies hard core contacts at various $r$ and thus broadens the peak. One can recover a sharp peak by constructing $g(\hat{r})$, with $\hat{r}_{ij} = r_{ij}/(R_i+R_j)$. In Fig.~\ref{fig:sizing}b we successfully used the sizes measured by our method to rectify the first peak.

Recently \citet{Kurita2011,Kurita2011b} have designed a sizing method using particle coordinates from confocal experiments. However their method do not work at the image processing level and relies on coordinates extracted via the \citet{Crocker1996} algorithm which, as described above, is defective when the size distribution is too broad. It may be possible to combine the two methods by feeding our coordinates and size as input to their method. We would expect an increase in sizing precision when the particles are close to contact.

%\bibliographystyle{naturemag3}
\bibliographystyle{apsrev4-1}
\bibliography{ico_dyn}

\end{document}